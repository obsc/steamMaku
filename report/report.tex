\documentclass{article}[12pt]
\usepackage{amsmath}
\usepackage{graphicx}
\usepackage{wrapfig}
\usepackage{framed}
\usepackage{tikz}
\usepackage{enumerate}
\usepackage[margin=2.5cm]{geometry}
\begin{document}
\begin{framed}
\noindent
\large{\textbf{CS 3110: Functional Programming \hfill Problem Set 6\\Bo Yuan Zhou (bz88), Rene Zhang (rz99)}}
\end{framed}

\noindent
\large{\textbf{Summary}}
\hspace*{\fill}

\hspace*{\fill}\\[\baselineskip]
\Large{\textbf{Instructions}}
\hspace*{\fill}
\begin{enumerate}
	\item Cd into the src directory
	\item Compile the project: Make
	\item Run the game server in one terminal: game/game.exe
	\item Open the GUI in another terminal: java -jar gui/gui\_client.jar
	\item Click on the "Connect" button in the GUI.
	\item Open a terminal and run a bot, which will be the red bot: team/babybot.exe
	\item Open a terminal and run a bot, which will be the blue bot: team/babybot.exe
\end{enumerate}
\hspace*{\fill}\\
\Large{\textbf{Design and Implementation}}
\hspace*{\fill}
\begin{itemize}
	\item \large{\textbf{Modules}}: 
	\item \large{\textbf{Architecture}}: 
	\item \large{\textbf{Code Design}}: 
	\item \large{\textbf{Implementation}}: 
\end{itemize}
\hspace*{\fill}\\
\Large{\textbf{Testing}}
\hspace*{\fill}

\hspace*{\fill}\\[\baselineskip]
\Large{\textbf{Extensibility}}
\hspace*{\fill}
\begin{itemize}
	\item New bullet types: \\
					In our Bullet.spawn we use the defined attributes of a bullet type (speed and radius) to determine its behavior. In order to create new bullet types, we would have to account for the new bullet type's attributes in our definitions and constants and then add new a section in our match statement in Bullet.spawn that creates a bullet with the behavior that we want.
	\item New types of collectible items: \\
					We have a function Powerup.spawn that we use to create powerups that both players can receive and a function Powerup.playerEvent which determines the effect the item has on a player. To create new collectible item types, we would create a match statement like we did for Bullet.spawn that checks the type of the collectible being spawned in Powerup.spawn to determine the behavior and in Powerup.playerEvent to determine the effects.
	\item More interesting bomb effects: \\
					% Stuff goes here
	\item Neutral enemies that fire at both players: \\
					Since we have a function Npc.spawn that creates npcs with a type of behavior (which defines what they do on a time tick), we can modify our Npc.spawn like we did Bullet.spawn so that we can take different Npc types and then create Npcs with different types of behavior using a match statement.
\end{itemize}
\hspace*{\fill}\\
\Large{\textbf{Known Problems}}
\hspace*{\fill}

\hspace*{\fill}\\[\baselineskip]
\Large{\textbf{Comments}}
\hspace*{\fill}
\end{document}